\documentclass{article}
\usepackage{ctex}
\usepackage{amssymb}
\usepackage{fancyhdr}
\usepackage{amsmath}
\usepackage{graphicx}
\usepackage{enumitem}
\usepackage{caption}
\small
\begin{document}
\pagestyle{fancy}
\lhead{田佳豪 3230105412}
\chead{数学软件期中项目作业报告}
\rhead{\today}
\section{Mandelbrot集介绍}
\subsection{题目背景}
Mandelbrot集是数学中的一个著名分形集合,它是由法国数学家Benoit Mandelbrotz在20世纪70年代提出的,
后来加以推广,成为复动力系统领域的重要研究对象。
\subsection{Mandelbrot集的定义}
Mandelbrot集是由复平面上的点组成的的集合,且
对于$c \in \mathbb{C} $ ,如果用迭代公式$ z_{n+1} = z_n^2 + c $ 进行迭代,其中$ z_0 = 0 $,
如果无论迭代多少次,$ z $的模长都不超过某个值,通常为2,则由所有的$ c \in \mathbb{C} $ 组成的集合成为Mandelbrot集。
\subsection{Mandelbrot集的性质}
\subsubsection{Mandelbrot集的图形}
整体上,Mandelbrot集主体在复平面上呈现出一种侧躺着的心形,而在其边缘上呈现或大或小螺纹状的突起,在这些突起上
仍然会产生更小的突起分支。
\subsubsection{Mandelbrot集的特征}
Mandelbrot集作为分形具有一些分形特征:
\begin{itemize}
    \item 自相似性,无论放大多少,边界都会是与整体相似的结构。
    \item 高复杂性,Mandelbrot集具有复杂的分形结构,无法用简单的几何图形来描述。
    \item 连通性,现已证明Mandelbrot集是连通的
\end{itemize}
\section{Mandelbrot集的C语言实现}
\subsection{算法思路}
我在mandelbrot.h文件中声明了mandelbrot结构体,并定义了初始化函数和图片展现函数。在.c文件中则调用这两个函数实现项目目的。
test.h文件则作为测试程序调用.h中的函数生成$(-1.5,-1),(1.5,1)$为端点的mandelbrot集图像。
\subsection{算法实现}
\subsubsection{初始化函数}
\begin{itemize}
    \item 输入参数:$x_{min},x_{max},y_{min},y_{max},max_{iter},width,height$作为生成图像的范围和尺寸,$max_{iter}$为迭代次数的上限。
    \item 然后分配行内存和列内存
    \item 之后根据迭代公式开始迭代,记录每个点的迭代次数,此处使用OpenMP并行计算。
\end{itemize}
\subsubsection{图片展现函数}
在此函数中,我将mandelbrot集的图像以bmp位图的方式保存,增加了展示效果,具体实现过程如下:
\begin{itemize}
    \item 首先在.c文件中声明一个bmp文件头,并在后面的函数中写入bmp文件数据。
    \item 然后根据迭代超过限定次数与否,设置绿色和红色,并将像素点写入到bmp文件中。
\end{itemize}
\subsubsection{测试程序}
在test.c文件中,我调用了.h中的函数生成$(-1.5,-1),(1.5,1)$为端点的mandelbrot集图像。之后调用图片展现函数将生成的图像保存为bmp文件。
最后释放原先分配的内存,完成静态图片测试。

然后使用迭代算法,取一个迭代中心点,不断缩小范围,用bmp文件保存每次迭代的结果,生成mandelbrot集动态视频。
\section{mandelbrot集的可视化展现}
\begin{figure}[htbp]
    \centering
    \includegraphics[scale=0.5]{mandelbrot.png}
    \caption{mandelbrot集图像}
    \label{fig:mandelbrot}
\end{figure}

\end{document}